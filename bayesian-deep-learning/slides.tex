\documentclass{beamer}
\usefonttheme[onlymath]{serif}
\usepackage{amsmath}
\usepackage{amsfonts}
\usepackage[export]{adjustbox}
\usepackage[utf8]{inputenc}
\usepackage{tikz} 
\usepackage{bm}
\usetikzlibrary{bayesnet}

% definitions
\def\H{\mathcal{H}}
\def\X{\mathbf{X}}
\def\w{\mathbf{w}}
\def\W{\mathbf{W}}
\def\const{\mathrm{const}}
\def\Var{\mathrm{Var}}
\def\tr{\mathrm{tr}}
\def\T{\top}
\def\U{\mathbf{U}}
\def\S{\mathbf{S}}
\def\V{\mathbf{V}}
\def\N{\mathcal{N}}
\def\E{\mathbb{E}}
\newcommand{\argmin}{\mathop{\mathrm{argmin}}}
\newcommand{\argmax}{\mathop{\mathrm{argmax}}}
\newcommand{\minimize}{\mathop{\mathrm{minimize}}}
\newcommand{\maximize}{\mathop{\mathrm{maximize}}}
\newcommand{\st}{\mathop{\mathrm{subject\,\,to}}}
\newcommand{\mat}[1]{\begin{bmatrix}#1\end{bmatrix}}

%Information to be included in the title page:
\usecolortheme{seahorse}
\title{Bayesian Deep Learning}
\author{Kangcheng Hou}
\institute{Zhejiang University}
\date{\today}

    
    
\begin{document}
    
\frame{\titlepage}

\begin{frame}{Bayesian framework}
Bayes theorem
$$\text{posterior} \propto \text{likelihood} \times \text{prior}$$
$$p(\mathbf{w} | \mathbf{X}, \mathbf{Y}) = \frac{p(\mathbf{Y} | \mathbf{X}, \mathbf{w}) p(\mathbf{w})}{p(\mathbf{Y} | \mathbf{X})}$$
Having the information of posterior distribution, we can predict an output for a new input point $\mathbf{x}^\ast$
$$p(\mathbf{y}^\ast | \mathbf{x}^\ast, \mathbf{X}, \mathbf{Y}) = \int p(\mathbf{y}^\ast |  \mathbf{x}^\ast, \mathbf{w}) p (\mathbf{w} | \mathbf{X}, \mathbf{Y}) d \mathbf{w}$$
An important component in Bayesian framework is model evidence,
$$p(\mathbf{Y} | \mathbf{X}, \mathcal{H}) = \int p(\mathbf{Y} | \mathbf{X}, \mathbf{w}, \mathcal{H}) p(\mathbf{w} | \mathcal{H}) d\mathbf{w}$$
It can be seen as marginalising the likelihood over $\mathbf{w}$.
\end{frame}

\begin{frame}{Variational Inference}
We are interested in the posterior distribution $p(\mathbf{w} | \mathbf{X}, \mathbf{Y})$, but this cannot usually be evaluated analytically. 

\end{frame}

\begin{frame}[allowframebreaks]{Bayesian Deep Learning}
Our target when developing is to make as less change to the current Deep Learning structure as possible and get uncertainty information from the model.
An important quantity in BDL is the posterior distribution of the weights.
$$p(\mathbf{w} | \mathbf{X}, \mathbf{Y})$$
Using variational inference, we use a parametrized distribution $q_\theta(\mathbf{w})$ to approximate the posterior distribution $p(\mathbf{w} | \mathbf{X}, \mathbf{Y})$.
We minimizes the KL divergence
\begin{align*}
\text{KL}(q_\theta(\mathbf{w}) || p(\mathbf{w} | \mathbf{X}, \mathbf{Y})) & = \int q_\theta(\mathbf{w}) \ln \frac{ q_\theta(\mathbf{w})}{p(\mathbf{w} | \mathbf{X}, \mathbf{Y})} d\mathbf{w} \\ 
& \propto \int q_\theta(\mathbf{w}) \ln \frac{ q_\theta(\mathbf{w})}{p(\mathbf{w})} - \int q_\theta(\mathbf{w}) \ln p(\mathbf{Y} | \mathbf{X}, \mathbf{w}) d\mathbf{w} \\ 
& = \text{KL}(q_\theta(\mathbf{w}) || p(\mathbf{w})) - \int q_\theta(\mathbf{w}) \ln p(\mathbf{Y} | \mathbf{X}, \mathbf{w}) d \mathbf{w}
\end{align*}
In DL context, this means the objective function is
$$\mathcal{L}_{\text{VI}} (\theta) = - \sum_{i=1}^N \int q_\theta(\mathbf{w}) \ln (p(\mathbf{y}_i | f^\mathbf{w} (\mathbf{x}_i))) d\mathbf{w} + \text{KL}(q_\theta(\mathbf{w}) || p(\mathbf{w}))$$
This objective function is not scalable because first the summed-over term is not tractable for DL model and $N$ terms is large in big data century. Using stochastic variational inference will give help with the large $N$ problem. And the monte carlo integration method will help us cope with the $$\int q_\theta(\mathbf{w}) \ln p(\mathbf{y}_i | f^\mathbf{w} (\mathbf{x}_i)) d\mathbf{w}$$ term.
\end{frame}
\begin{frame}[allowframebreaks]{Monte Carlo estimation for VI}
Now the problem is evaluating $$\int q_\theta(\mathbf{w}) \ln p(\mathbf{y}_i | f^\mathbf{w} (\mathbf{x}_i)) d\mathbf{w}$$ and optimize the quantity w.r.t $\theta$.
We consider a easier from of task
$$I(\theta) = \frac{\partial}{\partial \theta} \int f(x) p_\theta(x) dx$$
Here we use $p_\theta(x) = \mathcal{N} (x; \mu, \sigma^2)$ and $f(x)$ can be arbitrary. We introduce three methods:

\framebreak

\begin{block}{The score function estimator}
\begin{align*}
\frac{\partial}{\partial \theta} \int f(x) p_\theta(x) dx & = \int f(x) \frac{\partial}{\partial \theta} p_\theta(x) dx \\
& = \int f(x) \frac{\partial}{\partial \theta} \log p_\theta (x) p_\theta (x) dx 
\end{align*}
To estimate this quantity, we estimate $$\mathbb{E}_{x \sim p_\theta(x)}[f(x) \frac{\partial}{\partial \theta} \ln p_\theta](x)$$

\end{block}

\framebreak

\begin{block}{Reparametrisation trick}
If we can reparametrize the $p_\theta(x)$, for example, $p_\theta(x)$ can be $\mathcal{N}(\mu, \theta)$. We can thus reparametrize the $p_\theta(x)$ as $g(\theta, \epsilon) = \mu + \sigma \epsilon$ with $p(\epsilon) = \mathcal{N}(\epsilon; 0, I)$. We have the property that
$$p_\theta(x) dx = p(\epsilon) d \epsilon$$
\begin{align*}
\nabla_\theta \int f(x) p_\theta(x) dx & = \nabla_\theta  \int f(x) p(\epsilon) d\epsilon \\
& = \nabla_\theta  \int f(g(\theta, \epsilon)) p(\epsilon) d\epsilon \\ 
& = \mathbb{E}_{\epsilon \sim p(\epsilon)} [f(g(\theta, \epsilon))] \\ 
& = \mathbb{E}_{\epsilon \sim p(\epsilon)} [\nabla_\theta f(g(\theta, \epsilon))] \\
& = \mathbb{E}_{\epsilon \sim p(\epsilon)} [f'(g(\theta, \epsilon)) \nabla_\theta g(\theta, \epsilon)]
\end{align*}
\end{block}

Will not introduce the third method here.
\end{frame}

\begin{frame}[allowframebreaks]{Practical inference in Bayesian neural networks}
Previous work use fully fatorised Gaussian approximating distributions and characteristic function estimator in the approximation. This work will rely on the pathwise derivative estimator instead, so can make use of more interesting non-Gaussian approximating distributions. And to avoid losing weight correlations, fatorise the distribution for each weight row $\mathbf{w}_{l,i}$ in each weight matrix $\mathbf{W}_l$. With these two improvements, we can see that it is closely tied to SRT.

\framebreak

Note that we are going to optimize the quantity
$$\mathcal{L}_{\text{VI}} (\theta) = - \sum_{i=1}^N \int q_\theta(\mathbf{w}) \ln (p(\mathbf{y}_i | f^\mathbf{w} (\mathbf{x}_i))) d\mathbf{w} + \text{KL}(q_\theta(\mathbf{w}) || p(\mathbf{w}))$$
which can be optimized using batch information
$$\hat{\mathcal{L}}_{\text{VI}} (\theta) = - \frac{N}{M}\sum_{i\in S} \int q_\theta(\mathbf{w}) \ln (p(\mathbf{y}_i | f^\mathbf{w} (\mathbf{x}_i))) d\mathbf{w} + \text{KL}(q_\theta(\mathbf{w}) || p(\mathbf{w}))$$
We reparametrize each $q_{\theta_{l,i}}(\mathbf{w}_{l,i})$ as $\bm{w}_{l,i} = g(\theta_{l,i}, \epsilon_{l,i})$ with some specified $p(\epsilon_{l,i})$. We write $p(\bm{\epsilon}) = \prod_{l,i}(\epsilon_{l,i})$ and $\mathbf{w} = g(\theta, \bm{\epsilon})$. 

\framebreak

Thus our objective function is 
\begin{align*}
\hat{\mathcal{L}}_{\text{VI}} (\theta) & = - \frac{N}{M}\sum_{i\in S} \int q_\theta(\mathbf{w}) \ln p(\mathbf{y}_i | f^\mathbf{w} (\mathbf{x}_i)) d\mathbf{w} + \text{KL}(q_\theta(\mathbf{w}) || p(\mathbf{w})) \\
& = - \frac{N}{M}\sum_{i\in S} \int p(\bm{\epsilon}) \ln p(\mathbf{y}_i | f^{g(\theta, \bm{\epsilon})} (\mathbf{x}_i)) d\bm{\epsilon} + \text{KL}(q_\theta(\mathbf{w}) || p(\mathbf{w}))
\end{align*}
Replace the expected log likelihood term with its stochastic estimator, we get the MC estimator:
$$\hat{\mathcal{L}}_{\text{MC}} (\theta)  = - \frac{N}{M}\sum_{i\in S}  q_\theta(\mathbf{w}) \ln p(\mathbf{y}_i | f^\mathbf{w} (\mathbf{x}_i)) + \text{KL}(q_\theta(\mathbf{w}) || p(\mathbf{w}))$$
And 
$$\mathbb{E}_{S, \bm{\epsilon}} [\hat{\mathcal{L}}_{\text{MC}}] = \hat{\mathcal{L}}_{\text{VI}}$$
Optimizing $\hat{\mathcal{L}}_{\text{MC}}$ w.r.t $\theta$ would converge to the same optima as optimizing $\mathcal{L}_{\text{VI}}$.

\framebreak

To make predictive distribution, we do the following approximation:
\begin{align*}
p(\mathbf{y}^\ast | \mathbf{x}^\ast, \mathbf{X}, \mathbf{Y}) & = \int p(\mathbf{y}^\ast | \mathbf{x}^\ast, \mathbf{w}) p(\mathbf{w} | \mathbf{X}, \mathbf{Y}) d\mathbf{w} \\ 
& \approx \int p(\mathbf{y}^\ast | \mathbf{x}^\ast, \mathbf{w}) q_\theta(\mathbf{w})d\mathbf{w} \\ 
& \approx \frac{1}{T} \sum_{t=1}^T p(\mathbf{y}^\ast | \mathbf{x}^\ast, \hat{\mathbf{w}}_t)  \qquad \hat{\mathbf{w}}_t \sim q_\theta(\mathbf{w})
\end{align*}

\end{frame}

\begin{frame}[allowframebreaks]{Stochastic regularisation techniques}
Stochastic regularization techniques are techniques used to regularize deep learning models through the injection of stochastic noise into the model.

\end{frame}



\end{document}